\documentclass{jsarticle}
\begin{document}

\title{CPU実験最終レポート}
\author{1班 シミュレータ係 毛利公一 学生証番号05171031}
\maketitle

\section{自分の班のプロセッサについて}
\subsection{ISA}
RISC-Vを元に仕様を決めた。
\subsection{マイクロアーキテクチャ}
hoge
\subsection{シミュレータの機能}
hoge
\section{自分が担当した仕事について}
シミュレータ係を担当した。しかし、完動しておらず成果はまだない。そのため、途中経過を報告しそれを踏まえ今後やりたいことを述べようと思う。\\
まずアセンブリを読むシミュレータを作ろうとしたがsegmentation faultになって動かなかった。動かなかった理由は主に2つある。一つ目は仕様を理解していなかったからであり、二つ目は可読性が低くデバッグがしにくかったからである。具体的にはジャンプ命令の即値が仕様では相対アドレスであるが、絶対アドレスだと思い込んで実装していた。また、コピペを乱用したり、ファイル分割しなかったせいで読みにくくなった。\\
その後、コア係がシミュレータを改良し動くようにしてくれた。コア係のシミュレータとは別にシミュレータを作るため、機械語を読んで動くシミュレータを作ることにした。C言語のライブラリであるncursesをおすすめされたのでまずそれに触ってみた。自力でやろうとしたがよく分からなくなったため、先にできたシミュレータを写経し理解した。見やすくデバッグしやすいコードの書き方がかなり身についた。その後、自分で書いたが黒いppmファイルが出力されるだけでうまく動かなかった。反省点は仕様書などの与えられた資料を読まず先にできたコードを読んで理解しようとしたことと班員とあまり連絡を取らなかったことである。なぜなら、そのせいで効率の悪い作業になってしまったからである。\\
今後はまずデバッグの機能を増やし、シミュレータのバグをとりやすくすることを優先したい。また、余裕があれば結局使わずじまいだったライブラリncursesを利用して実装しtabや方向キーを使えるようにしたい。\\
\section{さらなる高速化に必要なプロセッサ・コンパイラの最適化と、それについての定量的な評価}
並列化

\end{document}
